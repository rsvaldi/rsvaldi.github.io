\documentclass[10.9pt]{res} % Use the res.cls style, the font size can be changed to 11pt or 12pt here
\renewcommand{\refname}{}
\usepackage{newcent} % To change the default font to the new century schoolbook postscript font uncomment this line and comment the one above

\newsectionwidth{0pt} % Stops section indenting

\usepackage{graphicx}
\usepackage[english]{babel}
\usepackage[applemac]{inputenc}
\usepackage{epstopdf}

%pacchetti matematici-math packages
\usepackage{amssymb}
\usepackage{amsthm}
\usepackage{amsfonts}
\usepackage{amsmath}
%\usepackage[all]{xy} 
\usepackage{amscd}
\usepackage{bbm}
\usepackage{hyperref}
\usepackage{url}
\usepackage{color}
%\usepackage{showkeys}
\numberwithin{equation}{section}

%thms & co
\newtheorem{theorem}{Theorem}%[section]
\newtheorem{proposition}[theorem]{Proposition}
\newtheorem{lemma}[theorem]{Lemma}
\newtheorem{definition}[theorem]{Definition}
\newtheorem{corollary}[theorem]{Corollary}
\newtheorem{def.thm}[theorem]{Definition-Theorem}
\newtheorem{question}[theorem]{Question}
\newtheorem{problem}[theorem]{Problem}
\newtheorem{conjecture}[theorem]{Conjecture}
\newtheorem{theoreman}[theorem]{Theorem $A_n$}
\newtheorem{theorembn}[theorem]{Theorem $B_n$}
\newtheorem{theoremcn}[theorem]{Theorem $C_n$}

\DeclareMathOperator{\cox}{Cox}

\theoremstyle{definition}
\newtheorem{remark}[theorem]{Remark}
\newtheorem{example}[theorem]{Example}
\newtheorem*{claim}{Claim}


\newcommand{\F}{\mathcal{F}}
\newcommand{\KF}{{K_\F}}
\newcommand{\NF}{{N_\F}}


\newcommand{\TX}{{T_X}}

\newcommand{\KG}{{K_\G}}
\newcommand{\KX}{{K_X}}
\newcommand{\KY}{{K_Y}}
\newcommand{\KZ}{{K_Z}}

\newcommand{\CNF}{{N^*_\F}}
\newcommand{\CNG}{{N^*_\G}}
\DeclareMathOperator{\kod}{kod}
%\input{../../../../Dropbox/latex/macros.ltx}

\makeatletter
\renewenvironment{thebibliography}[1]
     {\list{\@biblabel{\@arabic\c@enumiv}}%
           {\settowidth\labelwidth{\@biblabel{#1}}%
            \leftmargin\labelwidth
            \advance\leftmargin\labelsep
            \@openbib@code
            \usecounter{enumiv}%
            \let\p@enumiv\@empty
            \renewcommand\theenumiv{\@arabic\c@enumiv}}%
      \sloppy
      \clubpenalty4000
      \@clubpenalty \clubpenalty
      \widowpenalty4000%
      \sfcode`\.\@m}
     {\def\@noitemerr
       {\@latex@warning{Empty `thebibliography' environment}}%
      \endlist}
\makeatother

\begin{document}

\name{Roberto Svaldi\\ \\} % Your name at the top

% If you don't want one of the addresses, simply remove all the text in the first or second \address{} bracket

%\address{{\bf University of Cambridge}\\ DPMMS, Room C2.01, \\ Centre for Mathematical Sciences,\\ Wilberforce Road, \\Cambridge CB3 0WB, UK}

\address{{\bf \'Ecole polytechnique f\'ed\'erale de Lausanne } \\ 
EPFL SB MATH MATH-GE\\
MA B1 497 (B\^atiment MA)\\ 
Station 8\\
CH-1015 Lausanne, Switzerland}

\address{{\bf Personal data} \\ email: roberto.svaldi@epfl.ch\\ 
home page: \href{https://sma.epfl.ch/~svaldi/}{sma.epfl.ch/$\sim$svaldi/}\\
} % Your address 2
\begin{resume}

\section{\centerline{RESEARCH INTERESTS (for the general public)}}
\noindent\rule[0.5ex]{\linewidth}{.5pt}
\vspace{-20pt} % Gap between title and text


I specialize in algebraic geometry, with a particular interest and emphasis on the study of the 
birational structure of algebraic varieties and its applications to complex and analytic geometry. 

Algebraic geometry is a more than one hundred-year-old area of Mathematics. It has many connections to other branches, such as number theory 
and differential geometry. In recent years, algebraic geometry has seen substantially growing its applications to real 
life problems ranging from coding theory to computational biology. As a link between algebra and geometry, algebraic geometry attempts at
investigating algebraic objects using geometry and geometric objects using algebra. The algebraic objects considered are common zero 
sets of polynomials in any number of variables and their transformations are given by polynomial maps.
One of main the goals in the subject is to classify its objects of study, that are called algebraic varieties.
Their classification is a rather difficult task, yet a very important one. 
Its implications are not only limited to geometry per se, but they would find a wide range of applications in mathematics
and physics.\newline 
For example, theoretical physicists have shown a lot of interest for certain classes of algebraic varieties, 
called Calabi-Yau varieties, that naturally arise in the study of string theory. 
A better understanding of the geometry and the classification of Calabi-Yau varieties would give a fundamental advancement 
of string theory and would provide a lot of new examples and models to study.

There are two possible approaches to the classification of algebraic varieties:
\begin{itemize}
 \item on one hand, we can say that two distinct varieties are equivalent (or isomorphic)
 if we can stretch the shape of one of them into the shape of the other;
 \item on the other hand, we can say that two distinct varieties are birational equivalent (or simply, birational)
 if they both posses isomorphic big open sets.
\end{itemize}

When two algebraic varieties are birational, many numerical and geometrical quantities that capture their
shape are preserved. Hence, birational equivalence is a sufficiently coarse equivalence 
relation among geometrical objects. Birational equivalence can be detected at once if we have knowledge of an important algebraic 
object naturally attached to an algebraic variety: its field of rational functions.
Hence, this type of equivalence relation
is more flexible than just the classification by isomorphism type, as the function field of a variety
can be described quite easily. Moreover, we are allowed to change the shape of the variety to improve its features
as long as a huge chunk of it is left untouched: these operations will not change the field of rational functions 
and hence yield a new variety which is birational to the original one. 

Indeed, this is the leitmotif of the whole birational
classification process: among all varieties in a given birational class, we would like to find
one whose geometrical features are best possible. Such variety would be called a minimal model.

One of the main open problems in birational geometry is whether minimal models do exist.

Birational geometers have worked on this problem for about a century.
We know a sufficiently clear, but conjectural, picture of how the classification process
should work and what the outcomes should be.\newline
In fact, if we were able to actually construct minimal models, then it is not too hard to see that algebraic
varieties should be decomposable into three main types of ``building blocks''.
These are called:
\begin{description}
\item[log-Fano varieties] these are varieties which are very positively curved;
\item[Calabi-Yau varieties] these are varieties for which the curvature is flat, and
\item[log-canonical models] these are varieties which are very negatively curved.
\end{description}  
The classification scheme then proceeds with the construction of parameter spaces for 
the three building blocks above. Such parameter spaces are called moduli spaces.
A parameter space is in simple words a space parametrizing the possible shapes of given geometric objects.

Most of my work so far has revolved around three main themes: 
\begin{enumerate}
\item[(1)] Analysis of log-Fano varieties and Calabi-Yau varieties: recent results \cite{hmx.gen.type, BAB} have shown
that we should expect to be able to construct moduli parameter spaces for log-Fano and general type varieties.
I made some progress towards giving a positive answer in the Calabi-Yau case
in low dimension, \cite{mio.lcy}, which is a long-standing open problem.
This is the first type of result in this direction in the last 20 years.
In proving this result (in collaboration with G. Di Cerbo), we had to overcome several technical difficulties 
and we manage to establish new results on the boundedness of certain types of special fibrations.\newline
I am planning to generalize my results on Calabi-Yau varieties to any dimension
and thus establish a program to systematically study the possible types of
varieties that appear in this block.
\item[(2)] Study of existence and distribution of rational curves on algebraic varieties: 
rational curves are 2-dimensional spheres embedded in an algebraic variety in a very rigid (holomorphic) way.
Rational curves naturally appear when attempting to construct minimal models. 
Hence, it is natural to ask what are conditions that imply the existence or absence of rational curves on a given algebraic variety
and in the former case if and how it is possible to locate such curves. This is one of the leading themes of my research.
I have given in \cite{mio.hyperb} a new formulation of a classical result in birational geometry, the so-called Cone Theorem, that improves
our understanding of the position of rational curves with respect to a certain stratification induced by the most singular points of the variety.\newline
More generally, I am interested in studying the problem of existence of rational curves on algebraic varieties,
a very classical problem in algebraic geometry, that has been under speculation for a long time, cf. \cite{dem2012} for an account on this topic.
In \cite{mio.cy}, I show that on Calabi-Yau threefolds rational curves exist in a very large number of cases, 
solving almost completely a long-standing conjecture of Oguiso, \cite{Ogui93}.
\item[(3)] Application of techniques from birational geometry to algebraic foliations:
in collaboration with J. Pereira, see \cite{mio.fol.2}, we are able to apply the type of techniques that have been introduced to study the
problems from part (1) to obtain some new and long-awaited results in the theory of foliations -- special objects that are used to 
better understand the curvature of an algebraic variety. These results are completely new as we adopt
a different point of view on the subject, inspired by techniques in birational geometry, to rewrite the
now-classical results and show that certain numerical invariants can be effectively computed
and allow us to show that there are only finitely many types of cases that can happen.
\end{enumerate}


\newpage

\section{\centerline{RESEARCH INTERESTS (for mathematicians)}}
\noindent\rule[0.5ex]{\linewidth}{.5pt}
\vspace{-20pt} % Gap between title and text

My main research area is algebraic geometry with connections to complex and analytic geometry.
I am interested in the study of various aspects of the geometry of algebraic varieties: in particular, 
the classification of algebraic varieties,the study of the existence and distribution of special varieties -- e.g., rational curves --
and application of techniques from birational geometry to other fields.
%In the coming years, I plan to advance these subjects and their mutual connections and relations.
%Moreover, I plan on finding new ways to connect my research to other fields of mathematics.

Algebraic geometry is a one hundred-year-old area of Mathematics. It has many connections to other branches, such as number theory, 
commutative algebra and differential geometry. In recent years, algebraic geometry has seen its applications growing substantially  
in areas ranging from coding theory to computational biology. As a link between algebra and geometry, algebraic geometry attempts at
investigating algebraic objects using geometry and geometric objects using algebra. The algebraic objects considered are common zero 
sets of polynomials in more than one variable and their transformations are given by polynomial maps.

As it is often the case in geometry, one of main the goals in this field is to classify algebraic varieties. 
This is a rather difficult task. \newline
In dimension one, algebraic curves over the complex numbers are the same as compact Riemann surfaces. 
In dimension two, algebraic surfaces provide a rich class of examples of $4$-dimensional real manifolds that have been widely
studied in topology, differential and symplectic geometry and dynamics, to name a few.
In dimension three, for example, physicists have shown a lot of interest for certain classes of algebraic varieties, 
called Calabi-Yau varieties, that naturally arise in the study of string theory. 
A better understanding of the geometry and the classification of Calabi-Yau varieties would give a fundamental advancement 
of string theory and would provide a lot of new examples and models to study.

In dimension higher than 2 the classification is still very far from being satisfactory. 
The main issue is the lack of invariants that are sufficiently refined to distinguish among different types of objects. 
Rather than working with functions and maps that are defined everywhere on algebraic varieties, 
we may look at meromorphic functions and maps. Birational equivalence can be detected at once if we have knowledge of an important algebraic 
object naturally attached to an algebraic variety: the field of complex-valued functions which are locally defined as quotients of polynomials. 
Consider, for example, $\mathbb{C}$ equipped with ratio of polynomials $\frac{f(x)}{g(x)}$. \newline
The set of all meromorphic (or rational) functions over an algebraic variety is a field and gives an important invariant. 
Two varieties have isomorphic fields of rational functions if and only if they possess isomorphic
open dense subsets. In this case, we say that they are birational. Birational geometry is the study of algebraic varieties up to 
birational equivalence.

The classification of curves up to birational equivalence is the same as up to isomorphism. 
Already for surfaces, though, this is not true anymore. 
For example, consider $\mathbb{P}^2$ and $\mathbb{P}^1 \times \mathbb{P}^1$. They are birational as they contain a copy of 
$\mathbb{C}^2$, but they are not isomorphic as they have different topology.
The Italian school of algebraic geometry, in the first half of the 20th century, 
found a way to construct a preferred surface	 in a given birational equivalence class, whose geometric features are easy to study. 
These are named minimal surfaces. In the second half of the 20th century, a big effort in algebraic geometry has been dedicated to 
extending the ideas employed in the surface case to higher dimension.
\begin{conjecture}[Existence of minimal models]
\label{question.mmp}
It is possible to produce analogues of minimal surfaces for varieties of dimension higher than 2.
\end{conjecture}
\vspace{-8pt}
The Minimal Model Program (MMP), initiated in the '80's by Mori and carried out over the last 30 years by several other 
authors aims at solving Conjecture \ref{question.mmp} and using this result to classify algebraic varieties.

The MMP approach to the of classification varieties is divided into two steps, that are very different in principle, 
but are obtained by means of similar techniques, e.g., \cite{ueno.book.75}, \cite{kmm87},\cite{koll-etal-book}, 
\cite{koll-mor-book}, \cite{vieh.book.95}, \cite{kol.mod.13}, \cite{koll-book13}.\newline
First, one would like to prove the existence of minimal models and hence reduce the 
birational classification of algebraic varieties to the study of minimal models. \newline
What happens when a variety fails to be a minimal model? 
That happens when there are curves along which the tangent bundle is positively curved. 
The curvature of the tangent bundle governs the birational geometry of the variety.
For example, it is an important result of Miyaoka and Mori, \cite{miy.mor.unirul.86}, 
that if through a general point of a variety $X$ there is an algebraic curve on which the 
determinant of the tangent bundle is positive, then $X$ is uniruled, i.e. it is dominated by a product $\mathbb{P}^1 \times B$.
Uniruled varieties do not possess minimal models.
More in general, given a variety $X$ which is not a minimal model, in order to get a new variety birational to $X$ 
and closer to being a minimal model, one tries to contract the part of $X$ where 
the curvature has positivity. That part of $X$ contains lots of rational curves, i.e. curves isomorphic to $\mathbb{P}^1$. 
Hence, it is natural to ask what are conditions that imply the existence or absence of rational curves on a given algebraic variety
and in the former case if and how it is possible to locate such curves. This is one of the leading themes of my research.
I have given in \cite{mio.hyperb} a new formulation of the classical result in birational geometry, the so-called Cone Theorem, that improves
our understanding of the position of rational curves with respect to a certain stratification induced by the most singular points of the variety.
More generally, I am interested in studying the problem of existence of rational curves on algebraic varieties,
a very classical problem in algebraic geometry, that has been under speculation for a long time, cf. \cite{dem2012} for an account on this topic.
In \cite{mio.cy}, I show that on Calabi-Yau threefolds rational curves exist in a very large number of cases, 
solving almost completely a long-standing conjecture of Oguiso, \cite{Ogui93}.

Minimal models are expected to behave very nicely; 
namely, they should conjecturally be decomposable into two main building blocks: 
Calabi-Yau varieties and varieties of general type. A third ingredient, Fano varieties, is
needed to construct varieties which contain rational curves passing through a general point.
The classification would then proceed with the study of the geometry of three special types of varieties.
One of the fundamental questions related to this second step is how many types of 
Fano varieties, Calabi-Yau varieties and varieties of general type there are.
The ideal goal is to construct geometrical spaces that parametrize all possible
shapes for each given type, so-called moduli spaces. Recent results \cite{hmx.gen.type, BAB} have shown
that we should expect to be able to do that in the Fano and general type case.
I made some progress towards giving a positive answer in the Calabi-Yau case
in low dimension, \cite{mio.lcy}.
This is the first type of result in this direction in the last 20 years.
In proving this result (in collaboration with G. Di Cerbo), we have to overcome quite
a few technical difficulties and we manage to establish a foundational
result on the boundedness of certain types of special fibrations
that should have several applications also towards the so-called 
Manin's conjectures in arithmetic geometry, see \cite{leh1, leh2}.
I am planning to generalize my results on Calabi-Yau varieties to any dimension
and thus establish a program to systematically study the possible types of
varieties that appear in this block.
Moreover, I was able to apply the type of techniques that have been introduced to study this kind of 
problems to obtain some new and long-awaited results in the theory of foliations
in collaboration with J. Pereira, \cite{mio.fol.2}.

I am also interested in the geometry of log-Fano varieties, i.e., those varieties 
on which the curvature of the tangent bundle is positive everywhere, possibly apart from some subvariety of codimension at least $1$.
Log-Fano varieties appear naturally when a variety $X$ is covered by rational curves. 
Infact using the techniques of the Minimal Model Program, it is possible to find a birational model of $X$ 
which is fibered in log-Fano varieties. We still know very little about this class of varieties.
I have solved, in collaboration with others \cite{mio.num.toric}, a long-standing conjecture of Shokurov, bounding the number of
components of anticanonical divisors with nice singularities and their relation with toric varieties.






\newpage
\section{\centerline{\bf REFERENCES}}

\bibliographystyle{alpha}
\bibliography{all}

\end{resume} 
\end{document}