\documentclass[12pt]{article}


\usepackage{amscd,amssymb,amsfonts,amsmath}


\pagestyle{plain}

\begin{document}

\title{Mini-course: Linear systems on algebraic varieties.}
\author{Caucher Birkar (University of Cambridge)}
\date{}

\maketitle

In this short course, I plan to give an overview of recent progress in birational geometry and the classification theory of algebraic varieties.

Linear systems of divisors on varieties are fundamental in algebraic geometry, especially in birational geometry. 
Their asymptotic behaviour captures many of the essential properties of the ambient variety. 
One can associate various invariants to such systems to measure their complexity. 
One such invariant is the log canonical threshold which plays a central role in birational geometry and indicates 
how deep the singularities of the elements of the system are.

On the other hand, Fano varieties are of great interest in numerous parts of mathematics such as birational/algebraic geometry, 
differential geometry, arithmetic geometry, etc, due to the fascinating symmetries they possess.

I will discuss boundedness results regarding singularities of linear systems. 
Next I will explain how one applies these results to the case of Fano varieties 
to derive various boundeness properties of their anti-pluricanonical systems.
\end{document}