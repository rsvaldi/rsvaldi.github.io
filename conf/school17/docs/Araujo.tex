\documentclass[12pt]{article}


\usepackage{amscd,amssymb,amsfonts,amsmath}


\pagestyle{plain}

\begin{document}

\title{Mini-course: Fano foliations}
\author{Carolina Araujo (IMPA)}
\date{}

\maketitle

In the last few decades, much progress has been made in the classification of complex projective varieties.
The general viewpoint is that complex projective manifolds $X$ should be classified according to the behavior of their canonical class $K_X$.
As a result of the minimal model program, we know that every complex projective manifold can be built up from
3 classes of (possibly singular) projective varieties, namely,  varieties $X$ for which  $K_X$  satisfies
$K_X<0$, $K_X\equiv 0$ or $K_X>0$. 
Varieties $X$ whose anti-canonical class $-K_X$ is ample are called Fano varieties. 


In recent years, techniques from higher dimensional algebraic geometry, specially from 
the minimal model program, have been successfully applied to the study of global properties of 
holomorphic foliations. 
This led, for instance, to the birational classification of foliations by curves on surfaces.
The Fano condition naturally finds a counterpart in the theory of holomorphic foliations.
If $\mathcal{F}\subsetneq T_X$ is a foliation on a  complex projective manifold $X$, we define its canonical class to be 
$K_{\mathcal{F}}=-c_1(\mathcal{F})$. As in the case of projective manifolds,  numerical properties of $K_{\mathcal{F}}$
reflect geometric aspects of $\mathcal{F}$. 
We say that  $\mathcal{F}$ is a Fano foliation if $-K_{\mathcal{F}}$ is ample.

In a series of papers in collaboration with St{\'e}phane Druel (\cite{fano_fols}, \cite{codim_1_del_pezzo_fols}, \cite{fano_fols_2} and \cite{codim_1_mukai_fols}),  
we have carried out a systematic study of Fano foliations. 
In this mini-course, we shall explain the most important aspects of this theory, including the classification of Fano foliations of high index. 

\

\noindent {\bf Lecture 1.} Basic definitions and examples. Motivation: the problem of stability of the tangent bundle for Fano manifolds. 

\noindent {\bf Lecture 2.} Algebraicity criteria for foliations. Algebraicity for Fano foliations. 

\noindent {\bf Lecture 3.} Algebraically integrable foliations. The log leaf. 

\noindent {\bf Lectures 4 and 5.} Classification of Fano foliations of high index. Further developments, conjectures and open problems. 





\begin{thebibliography}{xxxxxx}


\bibitem{fano_fols}
{Carolina Araujo and St{\'e}phane Druel, \emph{On {F}ano foliations}, Adv. Math.
  \textbf{238} (2013), 70--118.}
  
\bibitem{codim_1_del_pezzo_fols}
{Carolina Araujo and St{\'e}phane Druel, \emph{On codimension 1 del {P}ezzo foliations on varieties with mild
  singularities}, Math. Ann. \textbf{360} (2014), no.~3-4, 769--798.}

\bibitem{fano_fols_2}
{Carolina Araujo and St{\'e}phane Druel, \emph{On {F}ano foliations 2}, in Foliation Theory in Algebraic Geometry, Proceedings of the conference "Foliation Theory in Algebraic Geometry", New York, NY, USA, September 3--7, 2013. Simons Symposia, p. 1-20, 2016.}

\bibitem{codim_1_mukai_fols}
{Carolina Araujo and St{\'e}phane Druel, \emph{Codimension one Mukai foliations on complex projective
  manifolds}, to appear in J. Reine Angew. Math. }

\end{thebibliography}

\end{document}